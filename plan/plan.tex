\documentclass[12pt,a4paper,english,oneside]{article}
\usepackage[utf8]{inputenc}
\usepackage{lmodern}
\usepackage[T1]{fontenc}
\usepackage{listings}
\usepackage{amsmath}
\usepackage{amssymb}
\usepackage{amsthm}
\usepackage[makeroom]{cancel}
\usepackage{mathtools}
\usepackage{hyperref}
\usepackage{changepage}
\usepackage{bm}
\usepackage{upgreek}
\usepackage{soul}
\usepackage{xcolor}
\newcommand\todo[1]{\textcolor{red}{TODO: #1}}


% Bibliography
\usepackage[
    backend=bibtex, % TODO change backend
    style=numeric,
    sorting=ynt
]{biblatex}
\addbibresource{refs.bib}

\emergencystretch=1em

\begin{document}

\title{Thesis Plan}
\author{Nikos Heikkilä}
\date{\today}
\maketitle

\section*{Background}

Distributed computing is any kind of computing that is performed on a spatially distributed system
\cite{DBLP:books/el/leeuwen90/LamportL90}.
Distributed algorithms are designed to solve distributed computing problems.
There are many different problems in distributed computing.
These problems are either possible or impossible to solve.
Problem can be solvable in one distributed computation model and non-solvable in others.
To show that a problem is possible or impossible to solve we need to have a proof.
It can be really tiresome to find proofs manually, therefore it is desirable to develope tools to automate proving (if possible).

If a problem is solvable, then there must exist an algorithm that solves the problem in any distributed network.
On the other hand if the problem is impossible to solve, then there must be a distributed network in which no algorithm can solve the problem.
In this work we are interested in the latter case, proving impossibility.

A distributed network is modelled as an undirected graph where each node represents a computer and each edge represents a connection between two computers.
Nodes are always connected to a fixed number of other nodes and the connections do not change.
A node that executes for constant time $t$ has to base its output solely on the information it has gathered from $t$-radius of other nodes.
\cite{DBLP:journals/siamcomp/NaorS95}

In locally checkable labeling (LCL) problems, a distributed algorithm produces labellings (a set of labels) for each node and edge.
Every node has a label for each of their incident edges, and similarly each edge has a label for each incident node
\cite{DBLP:journals/corr/abs-2105-05574}.
These labellings are used for locally checking the correctness of the solution.
A node's label for its incident edge needs to match with the corresponding label from the incident edge.
This must match for each label pair in order for the solution to be correct.
If there is one or more label pairs where the labels are different from each other, the whole solution becomes incorrect.

Researchers are interested in automating the detection of LCL-problem's constant time solvability in LOCAL model.
There already exists a tool called \emph{round eliminator} \cite{DBLP:conf/podc/Olivetti20} which uses a \emph{round elimination} \cite{DBLP:conf/podc/Brandt19} technique to automatically find lower bounds of LCL problems.
Round eliminator can tell if an LCL problem is solvable in constant time and this can be done fairly fast if the constant is small (as far as I know).
There can be multiple reasons for a LCL problem to not be solvable in constant time.
One such reason is that the problem is not solvable in PN (port number) model
\footnote{Some how there is a connection between constant time solvability in LOCAL model and solvability in PN model.}.
We would like to detect this particular case.


\section*{Objective}
The objective is to develope a tool (computer program) which tells if an LCL (locally checkable labelling) problem is not solvable in PN (port number) model.
The idea is to find a network where there exists no valid labellings implying that the problem is impossible to solve.
The tool should work fast, therefore optimizations are one of the main focuses of the work.

At first the tool will have a command line interface.
Later on the tool is planned to have a browser user interface.

The input should be given in the same format as the Round eliminator uses to represent LCL problems.
This is important because we can use the same input format in both tools.

Preferably the proof of non-solvability of the problem should be visualized.
This can be done for example by rendering the network that cannot have any viable configurations.

\section*{Schedule}

Working begins as soon as the plan gets approved.

\printbibliography

\end{document}
