%!tex root = ../main.tex

\section{Conclusion} \label{sec:conclusion}

I present a new algorithm that can detect if an LCL problem does not have a solution in finite connected $(\Delta, \delta)$-biregular multigraphs.
Then I show that if the problem does not have a solution in the graph family, it is also unsolvable in the PN model.

I also prove that if an LCL problem is not solvable in the PN model, it cannot be solved in constant time in the LOCAL model.
Thus, the algorithm can automatically prove that an LCL problem is not solvable in constant time in the LOCAL model i.e.\ giving it a lower bound of $\Omega(\log^* n)$.

In order to automatically find new lower bounds in practice, I present an implementation of the algorithm.
With the implementation, I find new lower bounds for 9 LCL problems and as a consequence, the problem
\begin{align*}
    A&=\text{AAA BBC},\\
    P&=\text{AB AC BB CC},
\end{align*}
is now classified with a tight bound of $\Theta(\log^* n)$.
The implementation uses the state-of-the-art graph generation and SAT solving software, in addition to parallelization, to be able to classify large LCL problem classes.

\subsection{Future Work}

Concerning the performance of the implementation, there is still room for improvements.
The graph generation is a huge bottleneck and generating large number of graphs causes the implementation to terminate.
If this can be solved, the implementation could classify problems using higher degree multigraphs.
The implementation in its current state does not fully support high performance computing.
With a support for high performance computing, we could potentially find even more interesting results than what we have found.
We could also analyze more closely the problems and multigraphs in which the problems are unsolvable.
This could reveal some general patterns or ideas for more classifiers.
