%!tex root = ../main.tex

\section{Research questions} \label{sec:research_question}

There are LCL problems that cannot be solved in the PN model.
More specifically, there exist LCL problems such that no PN-algorithm can solve it in \emph{every} graph.
There can be multiple reasons and one case we are aware of is that there exists no valid labeling for a graph.
\begin{researchquestion} \label{research_question:1}
Given an LCL problem and a graph, how can we automatically detect the non-existence of a valid labeling for the graph?
\end{researchquestion}

This thesis' main theorem is that an unsolvability of an LCL problem in the PN model implies that the problem is also not solvable in constant time in the LOCAL model.
We want to know if we can prove it.
\begin{researchquestion} \label{research_question:2}
Can we prove that an unsolvability of an LCL problem in the PN model implies that the problem is also not solvable in constant time in the LOCAL model?
\end{researchquestion}

In case we find answers to Research questions \ref{research_question:1} and \ref{research_question:2}, we want to implement a tool that can automatically detect the cases from Research question \ref{research_question:1}.
With the proven theorem from Research question \ref{research_question:2}, the results from the tool can be derived to non-constant lower bounds in the LOCAL model.
There also exists a database of upper and lower bounds of LCL problems \cite{Tereshchenko2021}, and these bounds have been computed using various tools called \emph{classifiers}.
Our implementation would be one kind of classifier, and we want to know if it can find any new lower bounds i.e. lower bounds that are higher than what is currently known about these problems.
\begin{researchquestion} \label{research_question:3}
For which LCL problems can we determine new lower bounds in the LOCAL model, using our implementation?
\end{researchquestion}

We want the implementation to be optimized, with fast execution time.
\begin{researchquestion} \label{research_question:4}
How fast can we make our implementation?
\end{researchquestion}
