%!tex root = ../main.tex

\section{Introduction}  \label{sec:introduction}

Large problems often require large computational capacity.
Distributed computing is often considered when high amounts of computation power are required.
It is used to solve large scale problems that would otherwise be impractical to solve in a centralized system.
Distributed computing is \emph{any kind} of computing that is performed on a spatially distributed system
\cite{DBLP:books/el/leeuwen90/LamportL90}.

In this thesis, we will be studying the theoretical foundations of distributed computing.
Commonly, the research in the field focuses on distributed algorithms and complexity classes of distributed graph problems.
%Our main focus is on locally checkable labelling (LCL) problems, which is a certain class of distributed graph problems.
Our main focus is on a certain class of distributed graph problems, locally checkable labelling (LCL) problems.

A distributed algorithm is a program that is executed in a distributed system.
Different computation models are used as an abstraction of these distributed systems.
The models we are particularly interested in are the port-numbering (PN) model and the LOCAL model.
\todo{this paragraph somehow needs polishing}

%We find LCL problems interesting, because a global result of a computation can be checked locally by each node.
%If all nodes agree that their local neighborhood of nodes are looking valid, then the result is also a globally valid solution.

%We want to know how fast LCL problems can possibly be solved
To gain more knowledge of complexities of LCL problems in the models, we will be focusing on finding lower bounds for them.
Finding new lower bounds for some LCL problems can potentially help us to figure out a general rule on why these LCL problems have their specific lower bound.
Finding a lower bound requires some kind of proof of existence, and proving it manually can be tiresome, so instead we want to automate the process.

\subsection{Objective}
In this thesis, we will prove a theorem that an unsolvability of an LCL problem in the PN model implies that the problem is also not solvable in constant time in the LOCAL model.
Using this theorem, we will implement a tool that can automatically find a proof of unsolvability of an LCL problem in the PN model.
We especially want this tool to work efficiently, leveraging parallelism.
To conclude, our objective is to find new lower bounds for LCL problems efficiently.

\subsection{Thesis structure}
We structure this thesis so that the reading order is sequential.
Section~\ref{sec:background} gives a succinct theoretical background to the topic of this work.
After the theoretical background, we present our research question in more detail in Section~\ref{sec:research_question}.
In Section~\ref{sec:prior_work}, we overview the prior work related to this thesis.
We introduce our algorithm and also prove the theorem in Section~\ref{sec:algorithm}.
The algorithm is used in our implementation, which we discuss in Section~\ref{sec:implementation}.
The results from our thesis are presented in Section~\ref{sec:results}, and finally we conclude the thesis in Section~\ref{sec:summary}.
