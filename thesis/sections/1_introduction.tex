%!tex root = ../main.tex

\section{Introduction}  \label{sec:introduction}

Large problems often require large computational capacity.
Distributed computing is often considered when high amounts of computation power are required.
It is used to solve large scale problems that would otherwise be impractical to solve in a centralized system.
Distributed computing is \emph{any kind} of computing that is performed on a spatially distributed system
\cite{DBLP:books/el/leeuwen90/LamportL90}.

In this thesis, we will be studying the theoretical foundations of distributed computing.
Our main focus is on locally checkable labelling (LCL) problems...
%The idea of distributed computing is to locally solve a part of a global solution by communicating with other

\todo{fix}


\subsection{Objectives}
In this thesis, we aim to implement a tool that can automatically find lower bounds for LCL problems.
\todo{fix}

\subsection{Thesis structure}
We structure the thesis so that one can read the sections in sequential order.
Section \ref{sec:background} gives a succinct theoretical background to the \todo[orange][]{topic of this work.}
After the theoretical background, we discuss our research question \todo[red][]{(TODO or questions?)} in more detail in Section \ref{sec:research_question}.
In Section \ref{sec:prior_work} we discuss the prior work related to this thesis.


First it explains ... %TODO Update these when background is done

Section \ref{sec:implementation}
\todo{fix}