%!tex root = ../main.tex

\section{Introduction}  \label{sec:introduction}
% According to thesistemplate, this section should be 2-4 pages long.
% It should contain following things:
% – Tutkimuksen taustaa ja tutkimusaiheen yleisluonteinen esittely
% – Tutkimuksen tavoitteet
% – Pääkysymys ja osaongelmat
% – Tutkimuksen rajaus ja keskeiset käsitteet.
%TODO Revise this section
%TODO Add subsection about thesis structure briefly describing each main sections contents
The topic of this thesis is related to the field of distributed computing.
It is a field in computer science that studies computation in a distributed fashion.
Distributed computing is any kind of computing that is performed on a spatially distributed system
\cite{DBLP:books/el/leeuwen90/LamportL90}.

Distributed algorithms are designed to solve distributed computing problems.
There are many different problems in distributed computing.
These problems are either possible or impossible to solve.
A problem can be solvable in one distributed computation model and non-solvable in others.
To show that a problem is possible or impossible to solve we need to have a proof.
It can be really tiresome to find proofs manually, therefore it is desirable to develope tools to automate proving.

If a problem is solvable, then there must exist an algorithm that solves the problem in any distributed network.
On the other hand if the problem is impossible to solve, then there exists no algorithm that solves the problem in all distributed networks.
In this work we are interested in the latter case, proving impossibility i.e. proving negative results.

A distributed network is modelled as an undirected graph where each node represents a computer and each edge represents a connection between two computers.
Nodes are always connected to a fixed number of other nodes and the connections do not change.
\cite{DBLP:journals/siamcomp/NaorS95}

An algorithm is the program that is given to each node in the network for execution.
Nodes may also be given different inputs but the programs are identical in every node.
The performance of an algorithm is usually measured by the amount of communication rounds used by the program.
The reason for this is that the nodes have only a limited information available about their neighbourhood and in order to increase this information they will need to communicate.
Communication between nodes takes a considerable amount of time in comparison to the processing capabilities of a node, therefore the reasonable unit of measurement is time in communication rounds.
Communication round is a synchronised event in which each node can communicate with each of their neighbor.
A node that executes for constant time $t$ has to base its output solely on the information it has gathered from $t$-radius of other nodes
\cite{DBLP:journals/siamcomp/NaorS95}.

Informally, locally checkable labelling (LCL) problems are a family of graph problems for which the validity of a solution can be verified by checking the solution of each node's $\mathcal{O}(1)$-radius neighbourhood (local neighbourhood).
In case all local neighbourhoods are valid, then the whole solution is globally valid.
\cite{DBLP:conf/podc/BrandtHKLOPRSU17}
%\footnote{This paragraph does not contain the formal definition of LCL problems.}

Graph coloring is an example of an LCL problem.
In a vertex coloring problem each vertex $v \in V$ of graph $G = (V, E)$ has a label $l(v): V \rightarrow \mathbb{N}$.
The problem itself is that for each $(v, w) \in E$ it must hold that $l(v) \neq l(w)$ i.e. labels must be assigned so that no two adjacent vertices have the same label.
It is an LCL problem because the solution can be verified locally.
Let the graph represent a distributed network.
Now each node can verify the solution by gathering information about its local neighbourhood of radius 1 and after that, simply check that it does not share a label with any of its neighbors.

Researchers are interested in automating the detection of LCL-problem's constant time solvability in LOCAL model.
There already exists a tool called \emph{round eliminator} \cite{DBLP:conf/podc/Olivetti20} which uses a \emph{round elimination} \cite{DBLP:conf/podc/Brandt19} technique to automatically find lower bounds of LCL problems.
Round eliminator can tell if an LCL problem is solvable in constant time in LOCAL model and this can be done fairly fast if the constant is small.% (as far as I know).
There can be multiple reasons for a LCL problem to not be solvable in constant time.
One such reason is that the problem is not solvable at all in PN (port number) model.
\footnote{Some how there is a connection between constant time solvability in LOCAL model and solvability in PN model.}.
We would like to detect this particular case.


\subsection*{Objective} %TODO Change to 'Objectives' and explain all of the objectives.
The objective is to develope a tool (computer program) which tells if an LCL (locally checkable labelling) problem is not solvable in PN (port number) model.
The idea is to find a network where there exists no valid labellings implying that the problem is impossible to solve.
The tool should work fast, therefore optimizations are one of the main focuses of the work.

At first the tool will have a command line interface.
Later on the tool is planned to have a browser user interface.

The input should be given in the same format as the Round eliminator uses to represent LCL problems.
This is important because we can use the same input format in both tools.

Preferably the proof of non-solvability of the problem should be visualized.
This can be done for example by rendering the network that cannot have any viable configurations.

\subsection{Thesis structure}
The chapter \ref{sec:background} gives a succint theoretical background to the topic of this work.
First it explains ... %TODO Update these when background is done

The chapter \ref{sec:implementation}
