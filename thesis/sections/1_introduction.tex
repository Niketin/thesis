%!tex root = ../main.tex

\section{Introduction}  \label{sec:introduction}

Large problems often require large computational capacity.
Distributed computing is often considered when high amounts of computation power are required.
Distributed computing is used to solve large scale problems that would otherwise be impractical to solve in a centralized system.
Distributed computing is \emph{any kind} of computing that is performed on a spatially distributed system
\cite{DBLP:books/el/leeuwen90/LamportL90}.

In this thesis, we will be studying the theoretical foundations of distributed computing.
Our main focus is on locally checkable labelling (LCL) problems...
%The idea of distributed computing is to locally solve a part of a global solution by communicating with other

\todo{fix}


\subsection{Objectives}
In this thesis, we aim to implement a tool that can automatically find lower bounds for LCL problems.
\todo{fix}

\subsection{Thesis structure}
The section \ref{sec:background} gives a succint theoretical background to the topic of this work.
First it explains ... %TODO Update these when background is done

The section \ref{sec:implementation}
\todo{fix}