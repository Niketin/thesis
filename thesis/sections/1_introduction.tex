%!tex root = ../main.tex

\section{Introduction}  \label{sec:introduction}

Large problems often require large computational capacity.
Distributed computing is often considered when high amounts of computation power are required.
It is used to solve large scale problems that would otherwise be impractical to solve in a centralized system.
Distributed computing is \emph{any kind} of computing that is performed on a spatially distributed system
\cite{DBLP:books/el/leeuwen90/LamportL90}.

In this thesis, we will be studying the theoretical foundations of distributed computing.
Commonly, the research in the field focuses on distributed algorithms and complexity classes of distributed graph problems.
%Our main focus is on locally checkable labelling (LCL) problems, which is a certain class of distributed graph problems.
Our main focus is on a certain class of distributed graph problems, locally checkable labelling (LCL) problems.

A distributed algorithm is a program that is executed in a distributed system.
Different computation models are used as an abstraction of these distributed systems.
The models we are particularly interested in are the port number (PN) model and the LOCAL model.
\todo{this paragraph somehow needs polishing}

%We find LCL problems interesting, because a global result of a computation can be checked locally by each node.
%If all nodes agree that their local neighborhood of nodes are looking valid, then the result is also a globally valid solution.

%We want to know how fast LCL problems can possibly be solved
To gain more knowledge about complexities of LCL problems in the models, we will be focusing on finding lower bounds for them.
Finding a lower bound requires some kind of proof that it exists, and proving manually can be tiresome, so instead we want to automate the process.
In this thesis, we aim to implement a tool that can automatically find lower bounds for LCL problems.
Finding new lower bounds for some LCL problems can potentially help us to figure out a general rule why these LCL problems have the specific lower bound.
\todo{reason more why we want to do this}

\subsection{Objectives}
Our first objective in this work is to automate the process of finding a proof of a non-constant lower bound for an LCL problem in the LOCAL model, in case the problem is unsolvable in the PN model.
%It is not trivial that these two cases have a connection, so we will
asda \todo{asddasdas}
The second objective of this work is to optimize the tool so that we can execute the procedure efficiently
\todo{asddasdas}


\subsection{Thesis structure}
We structure the thesis so that one can read the sections in sequential order.
Section \ref{sec:background} gives a succinct theoretical background to the topic of this work.
After the theoretical background, we present our research question (\todo[red][or]{questions?}) in more detail in Section \ref{sec:research_question}.
In Section \ref{sec:prior_work}, we overview the prior work related to this thesis.
We introduce our algorithm in Section \ref{sec:algorithm}.
The algorithm is used in our implementation, which we discuss in Section \ref{sec:implementation}.
The results from our thesis are presented in Section \ref{sec:results} and finally we conclude the thesis in Section \ref{sec:summary}.
