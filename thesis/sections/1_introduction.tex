%!tex root = ../main.tex

\section{Introduction}  \label{sec:introduction}

Large problems often require large computational capacity.
Distributed computing is often considered when high amounts of computation power are required.
It is used to solve large scale problems that would otherwise be impractical to solve in a centralized system.
Distributed computing is \emph{any kind} of computing that is performed on a spatially distributed system
\cite{DBLP:books/el/leeuwen90/LamportL90}.

In this thesis, we will be studying the theoretical foundations of distributed computing.
Commonly the research in the field focuses on complexity classes of computational problems in distributed systems.
Different computation models are used as an abstraction of these distributed systems.
Our main focus is on locally checkable labelling (LCL) problems in the port numbering (PN) model and in the local model.
%LCL problems are interesting, because given a labelled graph, each node can check locally if the labelling is a valid solution to the problem.
We find LCL problems interesting, because a global result of a computation can be checked locally by each node.
If all nodes agree that their local neighborhood of nodes are looking valid, then the result is also a globally valid solution.

\subsection{Objectives}
To gain more knowledge about complexities of LCL problems in the models, we will be focusing on finding lower bounds for them.
Finding these lower bounds require some kind of proof, and proving manually can be tiresome, so instead we want to automate the process.
In this thesis, we aim to implement a tool that can automatically find lower bounds for LCL problems.

Finding new classification to LCL problems can help us figure out some more general way of determining the complexity class of a wider class of problems blablabla...
Classifying these problems can potentially result
\todo{reason why we want to do this}

\subsection{Thesis structure}
We structure the thesis so that one can read the sections in sequential order.
Section \ref{sec:background} gives a succinct theoretical background to the \todo[orange][]{topic of this work.}
After the theoretical background, we discuss our research question (\todo{Questions?}) in more detail in Section \ref{sec:research_question}.
In Section \ref{sec:prior_work} we discuss the prior work related to this thesis.


First it explains ... %TODO Update these when background is done

Section \ref{sec:implementation}
\todo{fix}