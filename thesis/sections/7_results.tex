%!tex root = ../main.tex

\section{Results} \label{sec:results}
%TODO Show all results that have been found while doing the research, like different problems that have now new lower bounds.

%\begin{figure}[H]
% \centering
% \includegraphics[]{example-image-duck}
% \caption{\todo{Illustration of k-lift using Theorem \ref{lem:lcl_unsolvability:from_klift_to_simple}}}
% \label{fig:duck2}
%\end{figure}

In this section, we will discuss the results from our implementation.
We answer to Research question~\ref{research_question:3} in Section~\ref{sec:results:improving_bounds} by finding new lower bounds for LCL problems that have already been classified prior this work.
In Section~\ref{sec:results:classifying_large_classes}, we answer to Research question~\ref{research_question:4} by finding new lower bounds for classes of LCL problems.
We use the implementation~\cite[commit \codee{46ab4cc917}]{NonconstantLclClassifier2022} in its current state as of the time of writing this thesis.
When we say \codee{bin}, we refer to the compiled binary of the implementation.
Throughout this section, we assume that the results are obtained in a modern computer with AMD Ryzen 9 3900X 12-core 24-thread CPU, and 32 GB of RAM.

We do not use the caching feature of our implementation, as fetching already generated problem classes and multigraphs from the disk is substantially faster than generating them.
We want to give the measurements without pre-generated data unless we target some specific problems, like in Section~\ref{sec:results:improving_bounds}.



%We will answer to Research questions 3 by using our implementation for various LCL problems and showing their
%
%and 4 by showing classifications
%
\subsection{Improving lower bounds of LCL problems}\label{sec:results:improving_bounds}
There are two LCL problem classes in the database, $(\Delta, \delta, \lambda) = (3,2,3)$ and $(\Delta, \delta, \lambda) = (3,2,2)$.
The problems in the class $(3,2,2)$ are already classified with tight deterministic bounds.
This is not the case in class $(3,2,3)$, therefore we try to find new deterministic lower bounds for the problems using our implementation.

We use the latest database dump~\cite{DatabaseDump} of the database~\cite{Tereshchenko2021,LclClassifierAalto,LclClassifierGithub}.
In the class $(3,2,3)$, there are 7735 unique LCL-problems after purging and normalizing the problems, as we can see from Table~\ref{tbl:lcl_problem_classes}.
The database also agrees with the number of problems.

We are interested in the problems that have a constant ($\mathcal{O}(1)$) deterministic lower bound and a non-constant upper-bound, as these are the only problems for which we can possibly find a better lower bound of non-constant i.e. $\Omega(\log^*n)$.
When we remove problems with non-constant deterministic lower bounds, we get 6538 problems with a constant deterministic lower bound.
Out of these problems, we need to remove problems with constant upper bound, as these problems already have a tight constant bound, i.e. $\Theta(1)$.
After the removal, we have only 66 problems left.
The problems can be fetched from the database to a file \codee{problems.txt} using the command \verb|bin fetch_problems 3 2 3 <database_address> > problems.txt|, assuming that there is a PostgreSQL database running in the address \verb|<database_address>| with the data from the dump.

% \begin{figure}[H]
% \centering
% \includegraphics[]{example-image-duck}
% \caption{\todo{Maybe add a figure that explains the upper and lower bounds we want the problems to have. }}
% \label{fig:duck2}
% \end{figure}

\lstset{
  basicstyle=\ttfamily,
  mathescape,
  columns=fixed,
  fontadjust=true,
  basewidth=0.5em
}

We execute our implementation using the command \codee{time} from the shell \codee{Zsh} to measure the elapsed real time.
The whole command is
\begin{lstlisting}
time bin find <$\text{n}_{\text{low}}$> <$\text{n}_{\text{high}}$> from_stdin < problems.txt
\end{lstlisting}
where $n_\text{low}$ and $n_\text{high}$ are substituted with smallest and highest number of vertices we want for the graphs, respectively.
We run the command as follows in Table~\ref{tbl:results:asd1}.

\begin{table}[H]
    \centering
    %\begin{adjustbox}{width={\textwidth},keepaspectratio}%
    \begin{tabular}{rrrrrr}
        \toprule
        &&& \multicolumn{2}{c}{Time (s)} \\
        \cmidrule{4-5}
        $n_\text{low}$ & $n_\text{high}$ & \# of new lower bounds & Generate graphs & SAT\\
        \midrule
        %&&&&& \multicolumn{2}{c}{Trees} \\
        %Classifier & Complete & Labels & Paths & Cycles &Rooted & Unrooted \\\midrule
        1  & 5  & 1 & 0.031 & 0.002\\
        6  & 10 & 7 & 0.030 & 0.004\\
        11 & 15 & 8 & 0.062 & 0.015\\
        16 & 20 & 7 & 0.140 & 0.080\\
        21 & 25 & 7 & 3.758 & 0.510\\
        26 & 30 & 7 & 137.5 & 3.980\\
        %\vdots & \vdots &\vdots&\vdots\\
        \bottomrule
    \end{tabular}
    %\end{adjustbox}
    \caption{%
    Executing the command multiple times with different vertex ranges gives us new lower bounds.
    We can also see how long it takes to generate multigraphs, and also how long it takes to encode and solve SAT problems.
    }
    \label{tbl:results:asd1}
\end{table}

The executions from Table~\ref{tbl:results:asd1} were all executed with an identical input problem set, therefore some problems can be as a result in multiple executions.
When we consider only the smallest graph, where the problem was found to be unsolvable, there are only 9 distinct problems left.
The problems are listed in Table~\ref{tbl:results:asd2}.

\begin{table}[H]
    \centering
    \begin{adjustbox}{width={\textwidth},keepaspectratio}%
    \begin{tabular}{rlllllll}
        \toprule
        &&& \multicolumn{2}{c}{Lower bound} \\
        \cmidrule{4-5}
        $n$ & $A$ & $P$ & Old & New & Upper bound\\
        \midrule
        %&&&&& \multicolumn{2}{c}{Trees} \\
        %Classifier & Complete & Labels & Paths & Cycles &Rooted & Unrooted \\\midrule
        5  & AAA AAB BBC BCC     & AB CC       & $\Omega(1)$ & $\Omega(log^*n)$ & $\mathcal{O}(\log n)$\\
        10 & AAA AAB ABC BCC     & AC BB       & $\Omega(1)$ & $\Omega(log^*n)$ & $\mathcal{O}(\log n)$\\
        10 & AAA AAB ABC BCC     & AC BB CC    & $\Omega(1)$ & $\Omega(log^*n)$ & $\mathcal{O}(\log n)$\\
        10 & AAA AAB ABC BCC CCC & AC BB       & $\Omega(1)$ & $\Omega(log^*n)$ & $\mathcal{O}(\log n)$\\
        10 & AAA AAB ABC CCC     & AC BB       & $\Omega(1)$ & $\Omega(log^*n)$ & $\mathcal{O}(\log n)$\\
        10 & AAA AAB BCC         & AC BB CC    & $\Omega(1)$ & $\Omega(log^*n)$ & $\mathcal{O}(\log n)$\\
        10 & AAA ABC BBC         & AB BB CC    & $\Omega(1)$ & $\Omega(log^*n)$ & $\mathcal{O}(\log n)$\\
        10 & AAA BBC             & AB BB CC    & $\Omega(1)$ & $\Omega(log^*n)$ & $\mathcal{O}(\log n)$\\
        15 & AAA BBC             & AB AC BB CC & $\Omega(1)$ & $\Omega(log^*n)$ & $\mathcal{O}(\log^* n)$\\
        %\vdots & \vdots &\vdots&\vdots&\vdots&\vdots&\vdots\\
        \bottomrule
    \end{tabular}
    \end{adjustbox}
    \caption{%
    New lower bounds for LCL problems from the class $(3,2,3)$.
    Here, the column $n$ tells the size of the smallest multigraph, in which the problem is unsolvable.
    The last column lists the current known upper bound from the database.
    }
    \label{tbl:results:asd2}
\end{table}

Next, in Table~\ref{tbl:results:asd3}, we list all the multigraphs in which we found the problems to be unsolvable.

\begin{table}[H]
    \centering
    %\begin{adjustbox}{width={\textwidth},keepaspectratio}%
    \begin{tabular}{ll|r|r|r|r|r|r}
        \toprule
        && \multicolumn{6}{c}{$n$} \\
        \cmidrule{3-8}
        $A$ & $P$ & 5&10&15&20&25&30\\
        \midrule
        %&&&&& \multicolumn{2}{c}{Trees} \\
        %Classifier & Complete & Labels & Paths & Cycles &Rooted & Unrooted \\\midrule
        AAA AAB BBC BCC     & AB CC       & 1 &   &          &     &     &    \\
        AAA BBC             & AB AC BB CC &   &   & $9$      &     &     &    \\
        AAA AAB ABC BCC     & AC BB CC    &   & 1 & $5,9,12$ & $X$ & $Y$ & $Z$\\
        AAA AAB ABC BCC     & AC BB       &   & 1 & $5,9,12$ & $X$ & $Y$ & $Z$\\
        AAA AAB ABC BCC CCC & AC BB       &   & 1 & $5,9,12$ & $X$ & $Y$ & $Z$\\
        AAA AAB ABC CCC     & AC BB       &   & 1 & $5,9,12$ & $X$ & $Y$ & $Z$\\
        AAA AAB BCC         & AC BB CC    &   & 1 & $5,9,12$ & $X$ & $Y$ & $Z$\\
        AAA ABC BBC         & AB BB CC    &   & 1 & $5,9,12$ & $X$ & $Y$ & $Z$\\
        AAA BBC             & AB BB CC    &   & 1 & $5,9,12$ & $X$ & $Y$ & $Z$\\
        %\vdots & \vdots &\vdots&\vdots&\vdots&\vdots&\vdots\\
        \bottomrule
    \end{tabular}
    %\end{adjustbox}
    \caption{%
    List of multigraphs in which the problems are unsolvable.
    The multigraphs are listed as identifiers under each graph size $n$, and these numbers are specific to the size.
    The identifiers under sizes $20$, $25$ and $30$ are too large to be shown in this table, thus we use sets $X,Y,Z$.
    The sizes of these sets are $|X|=12, |Y|=66$ and $|Z|=424$.
    }
    \label{tbl:results:asd3}
\end{table}

We visualize the multigraphs of sizes $n \in \{5,10,15\}$ from Table~\ref{tbl:results:asd3}.
These graphs can be seen in Figure~\ref{fig:results:graphs}.

\begin{figure}[H]
    \subcaptionbox{
        $n=5$, $id=1.$
      \label{fig:results:graphs:n5_1}
    }%
    {
      \centering
      \includegraphics[scale=0.4]{diagrams/results_multigraph_n5_1.pdf}
    }
    \hfill
    \subcaptionbox{
        $n=10$, $id=1.$
      \label{fig:results:graphs:n10_1}
    }%
       {
      \centering
      \includegraphics[scale=0.4]{diagrams/results_multigraph_n10_1.pdf}
    }
    \hfill
    \subcaptionbox{
        $n=15$, $id=5.$
      \label{fig:results:graphs:n15_5}
    }%
       {
      \centering
      \includegraphics[scale=0.4]{diagrams/results_multigraph_n15_5.pdf}
    }
    \hfill
    \subcaptionbox{
        $n=15$, $id=9.$
      \label{fig:results:graphs:n15_9}
    }%
       {
      \centering
      \includegraphics[scale=0.4]{diagrams/results_multigraph_n15_9.pdf}
    }
    \hfill
    \subcaptionbox{
        $n=15$, $id=12$.
      \label{fig:results:graphs:n15_12}
    }%
       {
      \centering
      \includegraphics[scale=0.4]{diagrams/results_multigraph_n15_12.pdf}
    }
    \caption{All multigraphs from Table~\ref{tbl:results:asd3}, where $n \in \{5,10,15\}.$}
    \label{fig:results:graphs}
  \end{figure}


\subsection{Finding New Lower Bounds For Classes of LCL Problems} \label{sec:results:classifying_large_classes}
In this section, we try to find new lower bounds for various classes of LCL problems, with the focus on measuring performance.
We give the execution times of each execution similarly we did in Table~\ref{tbl:results:asd1}.


