%!tex root = ../main.tex
%%
%% Abstract in Finnish.  Delete if you don't need it.
%%
% LTeX: enabled=false
\thesistitle{Ei-vakioaikaisten alarajojen etsiminen automaattisesti paikallisesti tarkastettaville merkitsemisongelmille LOCAL-mallissa}
\supervisor{Prof.\ Jukka Suomela}
\advisor{Dr.\ Chetan Gupta}
%\degreeprogram{Elektroniikka ja sähkötekniikka}
%\department{Elektroniikan ja nanotekniikan laitos}
%\major{Sopiva pääaine}
%% The keywords need not be separated by \spc now.
\keywords{Paikallisesti tarkastettavat merkitsemisongelmat, LOCAL-malli, porttinumerointimalli, hajautetut algoritmit, hajautettu laskenta}
%% Abstract text
\begin{abstractpage}[finnish]
Hajautettu laskenta on mitä tahansa laskentaa, johon osallistuu eri paikoissa sijaitsevia tietokoneita.
Sitä käytetään usein tilanteissa, joissa vaaditaan suurta laskentatehoa.
Hajautettua laskentaa hyödyntämällä voidaan ratkaista ison mittakaavan ongelmia, jotka olisivat epäkäytännöllisiä ratkaista keskitetyssä järjestelmässä.

Tässä työssä tutkitaan hajautetun laskennan teoreettista perustaa.
Olen kiinnostunut hajautetun laskennan porttinumerointimallista eli PN-mallista sekä LOCAL-mallista.
Näissä malleissa jokainen tietokoneverkon solmu suorittaa samaa algoritmia synkronisesti ja vaihtaa jokaisella viestintäkierroksella viestejä viereisten solmujen kanssa.
Näiden viestintäkierroksien välillä solmut voivat laskea äärettömän nopeasti.
Algoritmin nopeus on se kierrosten määrä, jonka jälkeen jokainen verkon solmu viimeistään lopettaa algoritmin suorittamisen.
Paikallisesti tarkastettavat merkitsemisongelmat (LCL-ongelmat) ovat verkko-ongelmaperhe, jossa jokainen yksittäinen solmu voi paikallisesti tarkistaa globaalin ratkaisun.

Hajautetun laskennan teoreettisen perustan tutkimuksissa on viime aikoina näkynyt kehityssuunta, jossa automatisoidaan LCL-ongelmien ylä- ja alarajojen löytäminen LOCAL-mallissa.
Tämän työ edistää kyseistä tutkimusalaa automatisoimalla LCL-ongelmien ei-vakioaikaisten alarajojen löytämisen LOCAL-mallissa.

Esittelen uuden algoritmin, joka tunnistaa, jos annetulle LCL-ongelmalle ei ole ratkaisua äärellisissä yhtenäisissä $(\Delta, \delta)$-säännöllisissä kaksijakoisissa moniverkoissa.
Tämän jälkeen näytän, että jos tällä ongelmalla ei ole ratkaisua kyseisessä verkkoperheessä, niin se ei ole ratkeava PN-mallissa.
Näytän myös, että jos annettu LCL-ongelma ei ole ratkeava PN-mallissa, niin sitä ei voi myöskään ratkaista vakioajassa LOCAL-mallissa.
Tästä seuraa, että esittelemäni algoritmi voi todistaa automaattisesti, että LCL-ongelma ei ole vakioajassa ratkeava.
Esittelen myös toteutuksen algoritmille, jotta voimme käytännössä löytää automaattisesti uusia alarajoja LCL-ongelmille LOCAL-mallissa.
Löydän tällä toteutuksella uusia alarajoja yhdeksälle LCL-ongelmalle ja tämän seurauksena yhdelle näistä ongelmista tunnetaan sen laskennallinen vaativuus tarkasti: ongelman ratkaiseminen vaatii $\Theta(\log^* n)$ viestintäkierrosta.
\end{abstractpage}
% LTeX: enabled=true
