%!tex root = ../main.tex
%%
%% Abstract in Finnish.  Delete if you don't need it.
%%
% LTeX: enabled=false
\thesistitle{Ei-vakioaikaisten alarajojen etsiminen automaattisesti paikallisesti tarkastettaville merkitsemisongelmille LOCAL mallissa}
\supervisor{Prof.\ Jukka Suomela}
\advisor{Dr.\ Chetan Gupta}
%\degreeprogram{Elektroniikka ja sähkötekniikka}
%\department{Elektroniikan ja nanotekniikan laitos}
%\major{Sopiva pääaine}
%% The keywords need not be separated by \spc now.
\keywords{Paikallisesti tarkastettavat merkitsemisongelmat, LOCAL malli, porttinumerointimalli, hajautetut algoritmit, hajautettu laskenta}
%% Abstract text
\begin{abstractpage}[finnish]
Hajautettu laskenta on mitä tahansa laskentaa, jota suoritetaan tilallisesti hajautetussa järjestelmässä.
Sen käyttöä harkitaan usein, kun vaaditaan suurta laskentatehoa.
Hajautettua laskentaa hyödyntämällä voidaan ratkaista ison skaalan ongelmia, jotka olisivat muuten epäkäytännöllisiä ratkaista keskitetyssä järjestelmässä.

Tässä työssä tutkitaan hajautetun laskennan teoreettista perustaa.
Olen kiinnostunut hajautetun laskennan porttinumerointi- ja LOCAL-malleista.
Näissä malleissa, jokainen tietokoneverkon solmu suorittaa algoritmia synkronisesti ja vaihtaa jokaisella viestintäkierroksella viestejä viereisten solmujen kanssa.
Näiden viestintäkierroksien välillä solmut voivat laskea äärettömän nopeasti.
Algoritmin nopeutta mitataan viestintäkierroksien määrällä, kunnes jokainen verkon solmu lopettaa algoritmin suorittamisen.
\todo{edellinen lause kuulostaa oudolta}%TODO
Paikallisesti tarkastettavat merkitsemisongelmat (LCL) ovat verkko-ongelmaperhe, jossa jokainen yksittäinen solmu voi paikallisesti tarkistaa globaalin ratkaisun.

Esitän uuden algoritmin joka tunnistaa jos LCL-ongelmalle ei ole ratkaisua äärellisessä yhdistetyssä ()
\todo{jatka}
\end{abstractpage}
% LTeX: enabled=true
