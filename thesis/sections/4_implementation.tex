%!tex root = ../main.tex
%% Implementation
%%


%\section{Automatically proving lower bounds for LCL problems} \label{sec:implementation}
\section{Implementation} \label{sec:implementation}
In this chapter we will cover all the important topics that are used in the actual implementation.
The implementation is a tool, that attempts to automatically find a proof that shows the given LCL problem to be impossible to be solved in PN model (see section \ref{sec:port_number_model} for more information about PN model).

\subsection{The idea}
Given an LCL problem, the goal is to find a proof that it is impossible to solve the problem in PN model.
To prove this we can find a counter example that shows the impossibility.
A graph in which we cannot find any viable labelling is a good counter example, as this directly shows, that we cannot solve the LCL problem in every graph, therefore it is impossible to solve in PN model.

For each pair of LCL problem and graph, we want to be able to check if labelling is impossible.
For this purpose we can use for example a SAT solver.
This is indeed what we have used in this work.
%Probably, we have to iterate a lot of graphs.
%For this purpose we want to be able to generate the graphs.




\subsection{Generating LCL-problems}
\todo{parallelization}
\subsection{Generating graphs}
\todo{parallelization}
\subsection{SAT encoding and solving}
\todo{parallelization?}
%\subsection{Software optimizations}
\subsection{Caching}
\todo{pre-computing multigraphs}
\todo{pre-computing lcl problems (and power sets of lcl problems)}
\todo{caching}

\subsection{parallelization}
\todo{Talk about parallelization here or separately in above sections}

