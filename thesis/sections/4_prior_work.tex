%!tex root = ../main.tex

\section{Prior work} \label{sec:prior_work}
In this section, we discuss the related studies that have been released prior to this work.
First, we discuss the ... in Section \ref{sec:prior_work:title_a}.

\subsection{Meta computing and algorithm synthetization etc\todo{fix title}} \label{sec:prior_work:title_a}

Given that we have an interesting computational problem, we would like to learn something new about the computability of the problem in terms of computational complexity.
There are usually two desirable results that we would want to discover.
The first desirable result is an efficient algorithm that solves the computational problem.
The existence of such algorithm would show that the problem is solvable with \emph{at least} at the speed of the algorithm.
The second desirable result would be the information that such efficient algorithm does not exist at all.
If there are provably no algorithms with such efficiency, then possible algorithms have to be less efficient.
We say that the former results, showing existence, an upper bound, or possibility, are \emph{positive} results, and the latter results, showing nonexistence, a lower bound, or impossibility, are \emph{negative} results.
In our work, the objective is to find negative results, but we would also like to discuss finding positive results as it is the other side of discovering new information about distributed computational problems.%, and discussing gives better overall view on the subject.

Traditionally, positive and negative results for distributed computational problems have been found using the pen-and-paper method, for example This showed that there is a O(asd)-algorithm for the Problem \cite{} \cite{}.
The automation of finding these results is a more recent approach in the field \cite{} \cite{}.
%In this work we focus on the automation of new discoveries in the distributed computing.
In addition, there has been considerable work 

Proofs 
The first one, that we are targeting at in this thesis, is that we show the problem to not be solvable 
When we are attempting to 

There has been a lot of effort in showing positive results in the foundations


When we look at the 
From the bird's eye view, our work is one 

\subsection{Classifiers}

A \emph{classifier} is a tool that can automatically determine a lower bound or an upper bound for an LCL problem.

\todo{round eliminator, other classifiers (from aleksandr's lcl-classifier?)}
In our work, we focus only on showing negative results, i.e. we show that something is not possible to achieve, in a form of a lower bound.
In particular, we do


\subsection{Lower and upper bounds of LCL's}


0

%TODO I'll draft some sections here that I probably should consider writing about.
% Some might be really similar or even identical.
\subsection{\color{red}These sections are WIP, they do not necessarily stay this way, and each section probably does not stay under their own section}
\subsection{Lower and upper bounds of LCL's}
\subsection{Complexity classifications of LCL's}
\subsection{Computation of lower and upper bounds}
%TODO somewhere talk about the complexity classes, how there are infinitely many complexity classes, but in some models, range of complexity classes are equal. There are gaps in the complexity thingy.
\subsection{Computation in complexity theory}
\subsection{Complexity landscape of LCL's}
\subsection{Computer assisted something in complexity theory}
