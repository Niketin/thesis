%!tex root = ../main.tex

\section{Prior work} \label{sec:prior_work}
In Section~\ref{sec:prior_work:automation_of_proving}, we generally discuss the \emph{automation of finding} information about the best possible algorithm for a problem, and we briefly discuss what has been previously done on this topic in general.
In Section~\ref{sec:prior_work:automation_of_proving_in_local_model} we narrow our focus on the subject to the field of distributed computing.
In addition, we also discuss relevant automated tools that have been developed prior to this work and the complexity classes that are relevant in the LOCAL model.

\subsection{Automation of proving} \label{sec:prior_work:automation_of_proving}
Given any computational problem, we want to find the best possible (i.e. optimal) algorithm for the problem.
We like to learn something new about the computability of the problem in terms of computational complexity, in order to narrow the search for the best algorithm.
There are usually two things that we would want to discover about a problem.
\begin{itemize}
    \item
    An efficient algorithm that solves the problem.
    The existence of such algorithm would immediately give us an upper bound on the complexity of the problem.
    \item
    An efficient algorithm does not exist at all.
    Then we want to obtain a proof of nonexistence.
    This would give us a lower bound on the complexity of the problem.
\end{itemize}
Results, showing existence, an upper bound, or possibility, are called \emph{positive} results.
Similarly, the results, showing nonexistence, a lower bound, or impossibility, are called \emph{negative} results.
In our work, the objective is to find negative results, but we would also like to briefly discuss finding positive results as it is the other side of discovering new information about the computability of computational problems.

Traditionally, positive and negative results for computational problems are found using the pen-and-paper method.
For example, Linial\ \cite{DBLP:conf/focs/Linial87} showed in 1987 that there is a deterministic \(\mathcal{O}(\log^* n)\) algorithm for \(\mathcal{O}(\Delta^2)\)-coloring.
In a more recent paper from 2010, Barenboim and Elkin\ \cite{DBLP:conf/podc/BarenboimE10} showed that there is a deterministic \emph{polylogarithmic} time algorithm for \(\Delta^{1 + o(1)}\)-coloring, which is substantially better in terms of the number of colors, in comparison to the former one \cite{DBLP:conf/focs/Linial87} from Linial.

A more recent approach to finding positive or negative results is the \emph{automation} of the whole process.
One could automate the finding of positive or negative results by utilizing massive computational power of modern computers.
The case of negative results is exactly one of our work's objectives, and we will discuss the automation of finding these results in the distributed computing in Section \ref{sec:prior_work:automation_of_proving_in_local_model}.
But first, let us discuss the automation of finding positive results.

The automation of finding positive results involves creating a tool that takes a specification of desired behavior of an algorithm as an input, and outputs an algorithm that matches the specification.
This method is called \emph{algorithm synthesis}~\cite{DBLP:phd/basesearch/Rybicki16}.
For example, Dramnesc and Jebelean\ \cite{DBLP:journals/jsc/DramnescJ15} manage to automatically synthesize various sorting algorithms including selection sort, insertion sort, quick sort, merge sort and a novel sorting algorithm they call \emph{unbalanced merge sort}.
In another work, Gulwani et~al.\ \cite{DBLP:conf/pldi/GulwaniJTV11} present a tool that can efficiently synthesize highly nontrivial 10--20 line loop-free bit vector programs.

Propositional satisfiability solvers (SAT solvers) have been used in various synthesis problems to find positive results.
SAT solvers have been used especially in synthesis of circuits, where one goal is to find optimal Boolean circuits in terms of size.
For example, Järvisalo et~al.\ \cite{DBLP:conf/sat/JarvisaloKKK12} present a method to encode an ensemble computation problem as a propositional formula.
The encoding is then fed to a SAT solver to obtain either a working circuit design for the ensemble computation problem or a proof of nonexistence.
We use SAT solvers in our implementation to find negative results, and we will talk more about them in Sections~\ref{sec:implementation:sat} and~\ref{sec:implementation:our_sat}.


\subsection{Automation of proving in the LOCAL model} \label{sec:prior_work:automation_of_proving_in_local_model}


A tool that automates the finding of positive or negative results in distributed computing is called \emph{classifier}.
A classifier can automatically determine a lower bound or an upper bound for an LCL problem.
Currently, we are aware of only small number of classifiers in the LOCAL model, and we list them below in Table \ref{tbl:classifiers}.

Generally, there are infinitely many complexity classes possible because there are infinitely many functions.
However, in the LOCAL model, many of them turns out to be redundant.
\emph{Gap theorems} show that there are ranges of complexities where there exists no optimal algorithms for LCL problems.
For example, Balliu et~al.~\cite{DBLP:conf/podc/BalliuHOS19} showed that the deterministic complexity of an optimal algorithm in the LOCAL model is either at most \(\mathcal{O}(1)\) or requires at least \(\Omega(\log^* n)\) rounds.
In another work, Chang et~al.~\cite{DBLP:conf/focs/ChangKP16} showed that there is a gap between \(\mathcal{O}(\log^* n)\) and \(\Omega(\log n)\).
Currently, the complexity classes we care about, in the deterministic LOCAL model, are the following:
\[\mathcal{O}(1), \Theta(\log^* n), \Theta(\log n), n^{\Theta(1)}.\]
%This is the reason why the output complexity for an LCL problem, from a classifier, is typically one of these 4 complexities.

\begin{table}[H]
\centering
\begin{threeparttable}[t]
\begin{adjustbox}{width={\textwidth},keepaspectratio}%
\begin{tabular}{l l l l l l l}
    \toprule
    &&&&& \multicolumn{2}{c}{Trees} \\
    \cmidrule{6-7}
    Classifier & Complete & Labels & Paths & Cycles &Rooted & Unrooted \\\midrule
    %
    poly-classifier \cite{PolyClassifier2022, DBLP:journals/corr/abs-2202-08544}
    & Yes\tnote{1} & Any & No & No & Yes\tnote{2} & Yes\tnote{2} \\
    %
    Rooted tree classifier \cite{RootedTreeClassifier2021, DBLP:conf/podc/Balliu0OSST21}
    & Yes & Any & No & No & Yes\tnote{2} & No \\
    %
    Automata-theoretic lens classifier \cite{CyclePathClassifier2020, DBLP:conf/sirocco/ChangSS21}
    & Yes & Any & Yes & Yes & No\tnote{3} & No \\
    %
    Round Eliminator \cite{DBLP:conf/podc/Olivetti20, OlivettiRoundEliminatorGithub, DBLP:conf/podc/Brandt19}
    & No & Any\tnote{4} & Yes & No &  No & Yes \\
    %
    Tree-classifications (dataset) \cite{TreeClassifications2020}
    & No & 2, 3 & No & No & No  & Yes \\
    %
    TLP Classifier \cite{RocherTlpClassifier2020, Rocher2020}
    & No & 3 & Yes & No & No  & Yes \\ \bottomrule
    %
\end{tabular}%
\end{adjustbox}
\begin{tablenotes}[normal,flushleft]
    \footnotesize
    \item[1] Complete only in rooted trees.
    \item[2] Only in binary trees. In theory, it applies to all regular trees.
    \item[3] The paper~\cite{DBLP:conf/sirocco/ChangSS21} shows that it applies to some LCL problems in rooted trees.
    \item[4] Up to \(\frac{64}{\Delta}\).
\end{tablenotes}
\end{threeparttable}
\caption{A list of all classifiers we are aware of prior this work.} \label{tbl:classifiers}
\end{table}

In Table \ref{tbl:classifiers}, we list for each classifier along with the graph families they target to, the amount of labels they support and the completeness.
We define a classifier to be \emph{complete} if it can output a tight complexity class, from the previous list of 4 main complexities, for any input problem in the problem class it targets to.
There are multiple reasons for a classifier to be \emph{incomplete}, including:
\begin{itemize}
    \item it only outputs either upper or lower bound,
    \item it outputs ``no result'' i.e. it does not know the complexity class,
    \item it does not terminate.
\end{itemize}
Our classifier works only with unrooted trees and is incomplete in the sense that it gives only lower bounds, and only for some LCL problems.
In the table, we list the number of labels each classifier supports, and by `Any' we mean that a classifier supports reasonably high quantity of labels.
