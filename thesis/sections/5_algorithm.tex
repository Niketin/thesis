%!tex root = ../main.tex

\section{Algorithm} \label{sec:algorithm}

In this section, we will introduce the basic idea of an essential algorithm that is used in the actual implementation discussed in section \ref{sec:implementation}.
The algorithm's purpose is to find a proof that a given LCL problem $\Pi$ is impossible to solve in PN model.
Here we specifically allow PN networks to have multiple connections.
Later we show that the unsolvability of a problem $\Pi$ in PN networks (with multiple connections) implies that it is also impossible to solve $\Pi$ in \emph{simple} PN networks (without multiple connections).

\begin{definition} \label{def:lcl_solvability}
    An LCL problem $\Pi$ is solvable in PN model, if and only if there exists a PN algorithm $A$ that can find a solution for $\Pi$ in all PN networks.
\end{definition}

%The Definition \ref{def:lcl_solvability} works for PN networks with multiple connections\dots

We can define an alternative version of the Definition \ref{def:lcl_solvability} using contraposition.
\begin{definition} \label{def:lcl_solvability:contrapositive}
An LCL problem $\Pi$ \textbf{is not solvable} in PN model, if and only if there \textbf{does not exist} a PN algorithm $A$ that can find a solution for $\Pi$ in all PN networks.
\end{definition}

The Definitions \ref{def:lcl_solvability} and \ref{def:lcl_solvability:contrapositive} are equivalent. The latter one might be more difficult to understand, but the way it is phrased, easily leads us to the following Theorem:

\begin{theorem} \label{thm:lcl_nonsolvability}
    An LCL problem $\Pi$ is not solvable in PN model, if there exists a PN network $N$ such that no PN algorithm $A$ can solve the $\Pi$ in $N$.
\end{theorem}
\begin{proof}
    Let $\Pi$ be an LCL problem.
    Let $N$ be a PN network such that no PN algorithm $A$ can solve $\Pi$ in $N$.
    Therefore no PN algorithm $A$ can find a solution to $\Pi$ in all PN networks.
    According to the definition \ref{def:lcl_solvability:contrapositive}, this is equivalent with $\Pi$ being unsolvable.
\end{proof}

As the Theorem \ref{thm:lcl_nonsolvability} shows us, to show that a problem $\Pi$ is unsolvable, it is enough to find a counterexample, a PN network $N$ in which the problem $\Pi$ cannot be solved.
Using this fact, we can come up with an algorithm that automatically tries to find an counterexample.

\begin{algorithm}[H]
    \caption{Counterexample finder algorithm}
    \label{alg:counterexample_finder}
    \begin{algorithmic}[1] % The number tells where the line numbering should start
        \Require $1 \leq n_{low} \leq n_{high}$
        %\Require $\Pi$ is an LCL problem
        \Function{Find}{$n_{low},n_{high}, \Pi$} \Comment{Graph bounds $n_{low}$ and $n_{high}$, LCL problem $\Pi$} \label{alg:counterexample_finder:n_loop}
            \State $d_a \gets \textsc{ActiveDegree}(\Pi)$ \label{alg:counterexample_finder:d_a}
            \State $d_p \gets \textsc{PassiveDegree}(\Pi)$ \label{alg:counterexample_finder:d_p}
            \For{$n\gets n_{low}, n_{high}$} \Comment{Iterate graph sizes from lowest to highest} \label{alg:counterexample_finder:n}
                \State $G_n \gets \textsc{GenerateGraphs}(n, d_a, d_p)$ \label{alg:counterexample_finder:Gn}
                \ForEach{$g \in G_n$} \label{alg:counterexample_finder:g}
                    \If {$\textsc{IsUnsolvable}(\Pi, g)$} \label{alg:counterexample_finder:is_unsolvable}
                        \State \Return $g$ \label{alg:counterexample_finder:return_g}
                    \EndIf
                \EndFor
            \EndFor
            \State \Return \Comment{No counterexample found. Return nothing.} \label{alg:counterexample_finder:return_nothing}
        \EndFunction
    \end{algorithmic}
\end{algorithm}

The Algorithm \ref{alg:counterexample_finder} is designed to find the smallest counterexample graph for an LCL problem $\Pi$.
It starts from graphs of size $n_{low}$ and goes up to graphs of size $n_{high}$.
%TODO define what is a graph's size, in 2. background.
The variable for the current graph size is called $n$ (row \ref{alg:counterexample_finder:n}).
We define $d_a$ and $d_p$ to be the active and passive degree of problem $\Pi$ respectively (rows \ref{alg:counterexample_finder:d_a} and \ref{alg:counterexample_finder:d_p}).
For each graph size $n$, we generate all possible $(d_a, d_p)$-biregular multigraphs with $\textsc{GenerateGraphs}(n, d_a, d_p)$, and save the graphs into variable $G_n$ (row \ref{alg:counterexample_finder:Gn}).
Now, for each graph $g \in G_n$ (row \ref{alg:counterexample_finder:g}) we check if the given problem $\Pi$ is unsolvable using the function $\textsc{IsUnsolvable}(\Pi, g)$ (row \ref{alg:counterexample_finder:is_unsolvable}).
If it is unsolvable, we return the graph as an counterexample (row \ref{alg:counterexample_finder:return_g}).
In the case there are no counterexamples in graphs of size $n_{low}$ to $n_{high}$, the algorithm returns nothing (row \ref{alg:counterexample_finder:return_nothing}).

Up to this moment, we have not discussed about how the function $$\textsc{GenerateGraphs}(n, d_a, d_p)$$ from row \ref{alg:counterexample_finder:Gn} generates the graphs, nor have we discussed about how the function $$\textsc{IsUnsolvable}(\Pi, g)$$ from row \ref{alg:counterexample_finder:is_unsolvable} determines the unsolvability used in the Algorithm \ref{alg:counterexample_finder}.
These functions are implementation specific and in this section we assume that they exist as black boxes.
Later on the section \ref{sec:implementation} we introduce our implementations of these functions.
%First we would like to discuss about the $\textsc{GenerateGraphs}(n)$.
%As we know from the section \ref{}, %TODO refer to LCL section and make sure that it defines the LCL problems we are using here (LCL for (b,a)-biregular graphs).
%the LCL problems used in this thesis are defined for $(b,a)$-biregular graphs.
%Therefore we should not generate \emph{all} graphs but only the graphs of type $(b,a)$-biregular, where the degrees $b$ and $a$ are taken from the LCL problem $\Pi$.



%TODO talk about graphs, why they are actually (b, a)-biregular multigraphs.
%TODO talk about networks and graphs. In the above text they are refered as if they were the same thing.

\begin{theorem}
    Later we show that the unsolvability of a problem $\Pi$ in PN networks (with multiple connections) implies that it is also impossible to solve $\Pi$ in \emph{simple} PN networks (without multiple connections).
    \todo{Complete this theorem} % TODO complete this theorem
\end{theorem}

\begin{proof}
    We take a lift $N'$ from $N$ with where $\max \deg_N(v) * |N| =|N'|$.
    This new network $N'$ is by definition a simple network for which there exists a covermap $\phi(N')=N$.
    Thus there exists no PN algorithm $A$ that can solve problem $\Pi$ in $N'$.
    \todo{Complete this proof} %TODO Complete this proof
    %TODO Make sure that the covering maps and lifts are explained under section 2. background.
\end{proof}